For the calibration, my best solution was the cubic-spline interpolation, with which I got a per-line rms of 10.0 m/s. This is not of much help in the pursuit of RV precision to centimeters per second and nowhere near that of Excalibur, which, utilizing 1227 LFC exposures with various filtering of unstable lines and denoising techniques, achieves an rms of 3.3 cm/s \cite{zhao2021excalibur}. I used only one LFC exposure and performed no filtering. The spread of the residuals plotted in figure \ref{fig:calib_poly_vs_interp} is however not far from that of Zhao et al. (figure 2 in \cite{zhao2021excalibur}).

The results from analyzing the non-barycentric-corrected data were not entirely satisfactory. I measured one Earth-year to $\sim$366.5 days with a very small uncertainty. One day off. If EXPRES should be capable of measuring stellar RV to a precision of a dozen centimeters per second, I think I should be able to get this value right. However, considering the crudeness of my method and especially the resulting very large spread of the RVs computed for individual features (as seen in figure \ref{fig:median-mean-weighted-average}), it is perhaps not surprising, but rather justifies the decade-long development of much more advanced RV extraction methods.

Nevertheless, the signal is there. And, comparing my results for the barycentric-corrected data to that of Zhao et al. is also quite positive. RVs on the same order and many patterns in their results coincide with mine. The residuals from subtracting their results from mine are however quite large. For all four stars analyzed my rms also comes out smaller than theirs. This might be due to my match filtering, which is biased toward selecting features that are closer together if those features do not differ too much in shape.

As mentioned, the small errors could be due to the interpolation and sampling of extra data points. A way around this could be to interpolate not both peaks but only one. Instead of sampling new data points, I would just evaluate the interpolating function in the data-point locations of the un-interpolated peak and thereby have intensity values on a common wavelength range. As there might be some bias from interpolating only one of the peaks, one could then flip it around and interpolate the other one instead, finally taking the mean of the two results. This would lead to twice as many computations but also a lot fewer data-points. Depending on the overhead of iminuit I imagine it could even out. At a first attempt, however, I often got statistically incompatible results when flipping whose turn it was to be interpolated. Further investigation is necessary.

Another possible approach is to compute the radial velocity shift for each feature across all observations individually. This way one could also analyze the stability of each feature over time, as Zhao et al. do, and weight it accordingly, and one could perhaps even discover other patterns of stellar activity. Identifying and matching features across all observations might however be challenging, as, for instance due to stellar activity, some features might move significantly and overlap with others. If one is simply interested in the stable features however, this would be an easy way to filter out the bad ones. 

Having spent much more time during this project on exploring the data and devising solutions with my supervisor than reading papers, there is obviously much inspiration to gather from digging down more carefully in the literature. It is also conceivable that through the application of knowledge of stellar activity along with machine learning and a lot of data, features could be grouped and categorized according to the cause of their shift, which would be another way to sort out features that are unstable over time.

Finally, it is interesting to note that my solution and that of Zhao et al. do not yield completely overlapping results, although both solutions appear good. If two good, independent solutions do not give the same results, they must be extracting different information. It is therefore conceivable that a better solution exists, which extracts the combined information of both methods. That being said, there is no doubt, however, that their solution is better than mine and it is equally possible that the bias in my method is causing the deviation.