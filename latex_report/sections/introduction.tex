The radial velocity method was used to discover the first exoplanets and continues to be one of the main methods for the discovery and characterization of exoplanets today\cite{radial_velocity_techniques}. With new extreme-precision radial velocity (EPRV) spectrographs such as the EXtreme PREcision Spectrograph (EXPRES), data on which this project is based, we are slowly approaching the precision necessary for the discovery of Earth-sized planets around Sun-like stars. For this to succeed, it is however necessary to understand and mitigate many effects, such as the movement of the Earth relative to the center-of-mass of the solar system (barycentric corrections), light scattering in the Earth's atmosphere (tellurics), and light scattering inside the spectrograph (blaze). 
A general wavelength calibration of the spectrograph is also needed, the quality of which of course directly influences the precision of the final radial velocities that can be obtained. For that purpose, EXPRES utilizes a Laser Frequency Comb (LFC), which is a rather new technique. While to extract radial velocities (RV) we measure the apparent wavelength shift between observations over a period of time, preferably more than a year, and utilize the well-known Doppler effect. 

The full procedure from raw data to results, also referred to as the \emph{pipeline} in the literature, is extensive and complex, which is why I have ignored many aspects of it during this project. I have worked directly on the LFC calibration and RV extractions and tried to orientate myself about the most important corrections, thus that is what I will describe in this report. The full pipeline is described in detail in \cite{first_RV_from_EXPRES}. 